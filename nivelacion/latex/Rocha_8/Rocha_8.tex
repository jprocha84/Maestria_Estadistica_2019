\documentclass[12pt,letterpaper,fleqn]{article}
\usepackage{fullpage}
\usepackage[top=2cm, bottom=4.5cm, left=2.5cm, right=2.5cm]{geometry}
\usepackage{amsmath,amsthm,amsfonts,amssymb,amscd,mathtools}
\usepackage{lastpage}
\usepackage{enumerate}
\usepackage{fancyhdr}
\usepackage{mathrsfs}
\usepackage{xcolor}
\usepackage{graphicx}
\usepackage{listings}
\usepackage[unicode]{hyperref}
\usepackage{soul}
\usepackage[spanish]{babel}
\usepackage[utf8]{inputenc}
\usepackage{selinput}
\usepackage{enumitem}
\usepackage{amsmath}
\usepackage{textgreek}
\usepackage{newtxtext,newtxmath}
\SelectInputMappings{%
  aacute={á},
  ntilde={ñ},
  Euro={€}
}
\hypersetup{%
  colorlinks=true,
  linkcolor=blue,
  linkbordercolor={0 0 1}
}

\renewcommand\lstlistingname{Algorithm}
\renewcommand\lstlistlistingname{Algorithms}
\def\lstlistingautorefname{Alg.}

\lstdefinestyle{Python}{
    language        = Python,
    frame           = lines,
    basicstyle      = \footnotesize,
    keywordstyle    = \color{blue},
    stringstyle     = \color{green},
    commentstyle    = \color{red}\ttfamily
}

\setlength{\parindent}{0.0in}
\setlength{\parskip}{0.05in}

% Edit these as appropriate
\newcommand\course{Maestría en Estadística Aplicada}
\newcommand\hwnumber{1}                  % <-- homework number
\newcommand\NetIDa{Curso Nivelación - Ejercitación Nº8}           % <-- NetID of person #1
\newcommand\NetIDb{Juan Pablo Rocha Amado}           % <-- NetID of person #2 (Comment this line out for problem sets)

\pagestyle{fancyplain}
\headheight 35pt
\lhead{\NetIDa}
\lhead{\NetIDa\\\NetIDb}                 % <-- Comment this line out for problem sets (make sure you are person #1)
\chead{\textbf{\Large Unidad 6}}
\rhead{\course \\ \today}
\lfoot{}
\cfoot{}
\rfoot{\small\thepage}
\headsep 1.5em

\begin{document}

    \section*{Ejercitación Unidad 6}

    \begin{enumerate}[label=\textbf{\arabic*.}]
        \item % Ejercicio 1
        \begin{flalign*}
            \bar{x} &= 0.750\ mg \\
            s &= 0.175\ mg \\
            n &= 16\ cigarrillos
        \end{flalign*}
        \begin{enumerate}[label=\textbf{\alph*.}]
            \item Postule las hipótesis a ensayar.
            \begin{flalign*}
                H_0 &= \mu = 0.600\ mg \\
                H_1 &= \mu > 0.600\ mg \\
            \end{flalign*}

            \item Plantee la estadística de prueba a utilizar; justifique su elección.

            \item Calcule el p-value tome una decisión e interprete.

            \item Si se establece la siguiente regla de decisión: rechazar $H_0$ cuando el promedio observado sea 0.700 mg o mayor.

                \begin{enumerate}[label=\textbf{\arabic*.}]
                    \item Calcule la probabilidad de error tipo 1.

                    \item Calcule la probabilidad de error tipo 2, suponiendo que el promedio de nicotina postulado por la $H_1$ es 0.720.

                \end{enumerate}

            \item Construya un intervalo de confianza para poder dar respuesta al planteo de la organización desde otro enfoque.

        \end{enumerate}

    \end{enumerate}



\end{document}
