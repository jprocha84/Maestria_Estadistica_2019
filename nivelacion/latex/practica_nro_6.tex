\documentclass[12pt,letterpaper,fleqn]{article}
\usepackage{fullpage}
\usepackage[top=2cm, bottom=4.5cm, left=2.5cm, right=2.5cm]{geometry}
\usepackage{amsmath,amsthm,amsfonts,amssymb,amscd,mathtools}
\usepackage{lastpage}
\usepackage{enumerate}
\usepackage{fancyhdr}
\usepackage{mathrsfs}
\usepackage{xcolor}
\usepackage{graphicx}
\usepackage{listings}
\usepackage[unicode]{hyperref}
\usepackage{soul}
\usepackage[spanish]{babel}
\usepackage[utf8]{inputenc}
\usepackage{selinput}
\usepackage{enumitem}
\usepackage{amsmath}
\usepackage{textgreek}
\usepackage{newtxtext,newtxmath}
\SelectInputMappings{%
  aacute={á},
  ntilde={ñ},
  Euro={€}
}
\hypersetup{%
  colorlinks=true,
  linkcolor=blue,
  linkbordercolor={0 0 1}
}

\renewcommand\lstlistingname{Algorithm}
\renewcommand\lstlistlistingname{Algorithms}
\def\lstlistingautorefname{Alg.}

\lstdefinestyle{Python}{
    language        = Python,
    frame           = lines,
    basicstyle      = \footnotesize,
    keywordstyle    = \color{blue},
    stringstyle     = \color{green},
    commentstyle    = \color{red}\ttfamily
}

\setlength{\parindent}{0.0in}
\setlength{\parskip}{0.05in}

% Edit these as appropriate
\newcommand\course{Maestría en Estadística Aplicada}
\newcommand\hwnumber{1}                  % <-- homework number
\newcommand\NetIDa{Curso Nivelación - Ejercitación Nº6}           % <-- NetID of person #1
\newcommand\NetIDb{Juan Pablo Rocha Amado}           % <-- NetID of person #2 (Comment this line out for problem sets)

\pagestyle{fancyplain}
\headheight 35pt
\lhead{\NetIDa}
\lhead{\NetIDa\\\NetIDb}                 % <-- Comment this line out for problem sets (make sure you are person #1)
\chead{\textbf{\Large Unidad 4}}
\rhead{\course \\ \today}
\lfoot{}
\cfoot{}
\rfoot{\small\thepage}
\headsep 1.5em

\begin{document}

    \section*{Ejercitación Unidad 4}

    \begin{enumerate}[label=\textbf{\arabic*.}]

        \item % Ejercicio 1
            $\Omega = \{1C,1X,2C,2X,3C,3X,4C,4X,5C,5X,6C,6X \}$\\
            $R_x = \{ 2 ,1 ,3 ,2, 4, 3, 5, 4, 6, 5, 7, 6 \}$
            \begin{enumerate}[label=\textbf{\alph*.}]
                \item Obtenga la función de probabilidad $ p(x) $ \\ % a.
                $
                     p(x)=
                        \begin{dcases}
                          0 & 7<x<0 \\
                          \frac{1}{12} & x=1\\
                          \frac{1}{12} & x=7 \\
                          \frac{2}{12} & 2\leq x \leq 6
                      \end{dcases}
                $ \\
                \item Obtenga la función de distribución acumulada $F_X(x)$ \\ % b.
                    $ F(x) = $$\sum_{i=1}^{7} p(x_i) = \dfrac{1}{12} + \dfrac{2}{12} + \dfrac{2}{12} +\dfrac{2}{12}+\dfrac{2}{12}+\dfrac{2}{12}+\dfrac{1}{12}  = 1$$ $ \\

                \item Halle $P(X>3)$ \\% c.
                    $ P(X>3) = P(4)+P(5)+P(6)+P(7) = \boxed{\dfrac{7}{12}} $ \\

                \item Halle la probabilidad de que el puntaje obtenido sea un número impar. \\ %.d
                    C = \{x es impar\} \\
                    C = \{1,3,5,7\} \\
                    $ P(x) = P(1)+P(3)+P(5)+P(7) = \boxed{\dfrac{6}{12}} $
                    \\
                \item Halle $E(X), V(X)$ y $E(2X+3)$ \\ % e.
                    $ E(x) = $$\sum_{i=1}^{\infty} x_i p(x_i) = 1\dfrac{1}{12} +2\dfrac{2}{12} +3\dfrac{2}{12} +4\dfrac{2}{12} + 5\dfrac{2}{12}+  6\dfrac{2}{12}+  7\dfrac{1}{12} $$ = \boxed{4} $
                    \begin{flalign*}
                        V(x) & = E[X-E(X)]^2 \\&=
                        E(X^2) - [E(X)]^2  \\&=
                        1\dfrac{1}{12}+4\dfrac{2}{12}+9\dfrac{2}{12}+16\dfrac{2}{12}+ 25\dfrac{2}{12}+36\dfrac{2}{12}+49\dfrac{1}{12}-4^2  \\&=
                        \boxed{3.1667}
                    \end{flalign*}
                    $ E(2X+3) = 2E(X)+ 3 = \boxed{11} $
                    \\
            \end{enumerate}


        \item % Ejercicio 2
            \begin{enumerate}[label=\textbf{\alph*.}]
                \item Determinar el valor de la constante $ k $ \\ % a.
                    $ \int_{0}^{2} \int_{0}^{2} k(x+y) dx dy = 1 \\ $
                    \begin{flalign*}
                        \int_{0}^{2} k(x+y) dx & = \int_{0}^{2} kx dx + \int_{0}^{2} ky dx \\&=
                        \tfrac{kx^2}{2}\Big|_0^2 + kxy\Big|_0^2 \\&=
                        \frac{4k}{2} + 2ky \\&=
                        2k(y+1) \\
                    \end{flalign*}
                    \begin{flalign*}
                        \int_{0}^{2} 2k(y+1) dy & = \int_{0}^{2} 2ky dy  + \int_{0}^{2} 2k dy  \\&=
                        ky^2\Big|_0^2+2ky\Big|_0^2 \\&=
                        4k+4k \\&=
                        8k
                    \end{flalign*}
                    \begin{flalign*}
                        \int_{0}^{2} \int_{0}^{2} k(x+y) dx dy = 8k = 1 \\
                        \boxed{k = \frac{1}{8}}
                    \end{flalign*}
                \item Obtener las funciones de densidad marginales. \\ % b
                    \begin{flalign*}
                        f_x(X) & = \int_{0}^{2} \frac{1}{8}(x+y)dy \\&=
                        \int_{0}^{2} \frac{1}{8}x dy + \int_{0}^{2} \frac{1}{8}y dy \\&=
                        \frac{1}{8}xy\Big|_0^2 + \frac{1}{16}y^2 \Big|_0^2 =\\&=
                        \frac{1}{4}x + \frac{1}{4}\\&=
                        \boxed{\frac{1}{4}(x+1)}
                    \end{flalign*}
                    \begin{flalign*}
                        f_y(Y) & = \int_{0}^{2} \frac{1}{8}(x+y)dx \\&=
                        \int_{0}^{2} \frac{1}{8}x dx + \int_{0}^{2} \frac{1}{8}y dx \\&=
                        \frac{1}{16}x^2\Big|_0^2 + \frac{1}{8}yx^2 \Big|_0^2 \\&=
                        \frac{1}{4} + \frac{1}{4}y\\&=
                        \boxed{\frac{1}{4}(y+1)}
                    \end{flalign*}
                \item Calcular la $ P(X<1,Y\leq1.5),P(X\leq1),P(Y\leq1.5|X\leq1) $ \\ %c
                    \begin{flalign*}
                        F_x(x,y) &= \int_{0}^{x} \int_{0}^{y} f(x,y) dydx  \\&=
                        \int_{0}^{x} \int_{0}^{y} \frac{1}{8}(x+y)dydx
                    \end{flalign*}
                    \begin{flalign*}
                        \int_{0}^{y}\frac{1}{8}xdy + \int_{0}^{y}\frac{1}{8}ydy &= \frac{1}{8}xy\Big|_0^y + \frac{1}{16}y^2\Big|_0^y \\&=
                        \frac{1}{8}xy + \frac{1}{16}y^2 \\&=
                        \frac{1}{16}y(2x+y)
                    \end{flalign*}
                    \begin{flalign*}
                        \int_{0}^{x}\frac{1}{16}y(2x+y)dx & = \int_{0}^{x}\frac{1}{8}yxdx + \int_{0}^{x}\frac{1}{16}y^2dx \\&=
                        \frac{1}{16}yx^2\Big|_0^x + \frac{1}{16}y^2\Big|_0^x \\&=
                        \frac{1}{16}xy(x+y)
                    \end{flalign*}
                    \begin{flalign*}
                        P(X<1,Y\leq1.5) &= \frac{1}{16}xy(x+y) \\&=
                        \boxed{0.234375}
                    \end{flalign*}
                    \begin{flalign*}
                        P(X\leq1) &= \frac{1}{4}(x+1) \\&=
                        \boxed{0.5}
                    \end{flalign*}
                    \begin{flalign*}
                        P(Y\leq1.5|X\leq1) &= \dfrac{f(x,y)}{f_x(X)} \\&=
                        \dfrac{\dfrac{1}{16}yx(x+y)}{\dfrac{1}{4}(x+1)}\\&=
                        \boxed{0.46835}
                    \end{flalign*}
                \item ¿Son independientes $X$ e $Y$?  \\ %d
                    \begin{flalign*}
                        F_x(x,y) &= \dfrac{1}{16}xy(x+y) \\
                    \end{flalign*}
                    \begin{flalign*}
                        F_x(x) \cdot F_y(y) &= \dfrac{1}{4}(x+1) \cdot \dfrac{1}{4}(y+1) \\&=
                        \dfrac{1}{16}(xy+x+y+1) \\
                    \end{flalign*}

                    \fbox{\begin{minipage}{15em}
                        $ F_x(x,y) \neq F_x(x) \cdot F_y(y) $\\
                        X e Y no son independientes
                    \end{minipage}}
            \end{enumerate}

        \item % Ejercicio 3
            $ P(Y>X) = \cfrac{\lambdaup _1}{\lambdaup _1+\lambdaup _2} $ \\\\
            $ \int_{0}^{y} \int_{0}^{y} \lambdaup _1 \lambdaup _2 e^{-\lambdaup _1 x - \lambdaup _2y} dy dx $
            \begin{flalign*}
                \int_{0}^{y} \lambdaup _1 \lambdaup _2 e^{-\lambdaup _1 x - \lambdaup _2y} dy = \lambdaup _1(e^{\lambdaup _2y}-1) e^{-\lambdaup _1 x - \lambdaup _2y}
            \end{flalign*}
            \begin{flalign*}
                \int_{0}^{y} \lambdaup _1(e^{\lambdaup _2y}-1) e^{-\lambdaup _1 x - \lambdaup _2y} dx = (e^{\lambdaup _1 x}-1)(e^{\lambdaup _2 y}-1)e^{-\lambdaup _1 x - \lambdaup _2 y}
            \end{flalign*}
    \end{enumerate}



\end{document}
